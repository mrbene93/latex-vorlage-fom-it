% Präambel
\documentclass[
    a4paper,
    12pt,
    listof=totoc,
    toc=chapterentrydotfill
]{scrartcl}

% Allgemeine Pakete
\usepackage[utf8]{inputenc}                             % UTF-8, Umlaute
\usepackage[ngerman]{babel}                             % Deutsche Bezeichnungen
\usepackage[ngerman]{translator}                        % Deutsche Bezeichnungen
\usepackage{fancyhdr}                                   % Manipulation Kopf- & Fußzeile
\usepackage{graphicx}                                   % Unterstützung für Grafiken
\usepackage{color}                                      % Farbverwaltung
\usepackage[pdfborder={0 0 0}]{hyperref}                % Erzeugung von URLs
\usepackage{footnote}                                   % Verwaltung von Fußnoten
\usepackage{colortbl}                                   % Farben für Tabellen
\usepackage{lscape}                                     % Querformat
\usepackage{makeidx}                                    % Indexerstellung
\usepackage{ragged2e}                                   % Textausrichtung
\usepackage{verbatim}                                   % Kommentarumgebung
\usepackage{marvosym}                                   % Symbolschriftart
\usepackage[autostyle=true,german=quotes]{csquotes}     % Anführungszeichen-Stil
\usepackage[printonlyused]{acronym}                     % Abkürzungsverzeichnis
\usepackage{xpatch}                                     % Macro Patching
\usepackage[figure,table]{totalcount}                   % Zähler
\usepackage{calc}                                       % Arithmetische Befehle
\usepackage{tabularx}                                   % Benutzerdefinierbare Tabellen
\usepackage{booktabs}                                   % Tabellen

% Meta-Informationen
\newcommand{\metaTitel}{\LaTeX{} Vorlage für eine Hausarbeit gemäß der Richtlinien der FOM.}
\newcommand{\metaAutor}{Theodor Test}
\newcommand{\metaMatrikelNr}{123456}
\newcommand{\metaBetreuer}{Prof.\ Dr.\ Siegbert Schnösel}
\newcommand{\metaAbgabe}{01. Januar 1970}
\newcommand{\metaStandort}{Hochschulzentrum Essen}
\newcommand{\metaOrt}{Essen}
\newcommand{\metaStudiengang}{Bachelor of Science in Prokrastination}
\newcommand{\metaSemester}{12}
\newcommand{\metaModul}{Procrastination Basics}

% Formatierung
\usepackage[
    left=   4.0cm,
    right=  2.0cm,
    top=    4.0cm,
    bottom= 2.0cm
]{geometry}
\setlength{\parindent}{0mm}
\setlength{\parskip}{6pt plus 2pt minus 1pt}
\sloppy

% Schriftarten
\usepackage[T1]{fontenc}
\usepackage{fontspec}
\setmainfont[
    Path            = fonts/,
    Extension       = .ttf,
    UprightFont     = *-Regular,
    ItalicFont      = *-Italic,
    BoldFont        = *-Bold,
    BoldItalicFont  = *-BoldItalic
]{LiberationSerif}
\setsansfont[
    Path            = fonts/,
    Extension       = .ttf,
    UprightFont     = *-Regular,
    ItalicFont      = *-Italic,
    BoldFont        = *-Bold,
    BoldItalicFont  = *-BoldItalic
]{LiberationSerif}
\setmonofont[
    Path            = fonts/,
    Extension       = .ttf,
    UprightFont     = *-Regular,
    ItalicFont      = *-Italic,
    BoldFont        = *-Bold,
    BoldItalicFont  = *-BoldItalic
]{LiberationMono}
\usepackage[onehalfspacing]{setspace}

% Silbentrennung
\usepackage{hyphsubst}
\HyphSubstIfExists{ngerman-x-latest}{
\HyphSubstLet{ngerman}{ngerman-x-latest}}{}

% Counter
\setcounter{tocdepth}{3}

% PDF Metadaten
\hypersetup{
    pdfinfo={
        Title={\metaTitel},
        Subject={\metaStudiengang},
        Author={\metaAutor},
    }
}

% Kopfzeile
\pagestyle{fancy}
\fancyhf{}
\rhead{\thepage}

% Glossar
\usepackage[toc]{glossaries}
\makeglossaries{}

% Überschriften
\usepackage{caption}
\captionsetup{
    justification=raggedright,
    labelfont=bf,
    singlelinecheck=false,
    textfont=bf
}

% Sektionsnummerierungen ohne abschließenden Punkt
\renewcommand*{\autodot}{}

% Syntax Highlighting
\usepackage{listings}
\definecolor{light-gray}{gray}{0.95}
\definecolor{dark-gray}{gray}{0.4}
\lstset{
    backgroundcolor=\color{light-gray},
    basicstyle=\ttfamily\scriptsize,
    breakatwhitespace=false,
    breaklines=true,
    commentstyle=\color{dark-gray},
    extendedchars=true,
    keepspaces=true,
    keywordstyle=\color{blue},
    language=Python,
    numbers=none,
    numbersep=5pt,
    numberstyle=\scriptsize\color{dark-gray},
    showspaces=false,
    showstringspaces=false,
    showtabs=false,
    stringstyle=\color{red},
    tabsize=4
}

%% Literaturmanagement
\usepackage[
    autocite=footnote,
    backend=biber,
    bibstyle=ext-authoryear,
    citestyle=authortitle,
    dashed=false,
    date=year,
    dateabbrev=false,
    giveninits=true,
    innamebeforetitle=true,
    maxbibnames=999,
    maxcitenames=3,
    mergedate=false,
    mincrossrefs=1,
    seconds=true,
    uniquename=init,
    urldate=long
]{biblatex}
\addbibresource{literatur/literatur.bib}
% Fußnoten
\renewcommand{\multicitedelim}{;\space}
\DeclareNameAlias{labelname}{family-given}
\xapptobibmacro{cite}{\setunit{\nametitledelim}\printfield{year}}{}{}
% Literaturverzeichnis
\setlength\bibhang{1cm}
\setlength{\bibinitsep}{0.75cm}
\renewcommand*{\newunitpunct}{, }
\renewcommand*{\finentrypunct}{}
\renewcommand*{\UrlFont}{\ttfamily\small}
% Quelle unter Abbildungen
\newcommand*{\abbcite}[1]{\par\raggedleft\footnotesize Quelle:~\cite{#1}}
\newcommand*{\abbeig}{\par\raggedleft\footnotesize Quelle: Eigene Darstellung}



% Inhalt ------------------------------------------------------
\begin{document}
    \lhead{\nouppercase{\leftmark}}
    \pagenumbering{Roman}
    %TC:ignore

    % Titelblatt
    \begin{titlepage}
        \begin{center}
            \begin{large}
                \includegraphics[width=5cm]{abbildungen/fom-logo.png}\\
                \vspace{1.5cm}
                \textbf{FOM {--} Hochschule für Oekonomie und Management}\\
                \metaStandort{}\\
                \vspace{1.0cm}
                Berufsbegleitender Studiengang zum\\
                \metaStudiengang{}\\
                \vspace{1.0cm}
                \metaSemester{}. Semester\\
                Hausarbeit im Modul\\
                \textbf{\metaModul{}}\\
                zum Thema\\
            \end{large}
            \vspace{1.5cm}
            \begin{Large}
                \textbf{\metaTitel{}}
            \end{Large}
        \end{center}
        \vfill
        \begin{tabbing}
            Links \= Mitte \= Mittez \= Rechts \kill
            Betreuer: \> \> \> \metaBetreuer{}\\
            Autor: \> \> \> \metaAutor{}\\
            Matrikelnr.: \> \> \> \metaMatrikelNr{}\\
            Abgabedatum: \> \> \> \metaAbgabe{}
        \end{tabbing}
    \end{titlepage}
    \newpage

    % Inhaltsverzeichnis
    \tableofcontents
    \newpage

    % Abbildungsverzeichnis
    \listoffigures
    \newpage

    % Tabellenverzeichnis
    \listoftables
    \newpage

    % Abkürzungsverzeichnis
    \lhead{Abkürzungsverzeichnis}
    \section*{Abkürzungsverzeichnis}
\phantomsection{}
\addcontentsline{toc}{section}{Abkürzungsverzeichnis}

\begin{acronym}
    \acro{FOM}{(früher) Fachhochschule für Oekonomie und Management}
\end{acronym}

\newpage


    % Glossar
    \lhead{Glossar}
    \newglossaryentry{latex}
{
    name=TeX,
    description={
        Ein Textsatzsystem mit eingebauter Makrosprache, welches Quelltext einliest und mit ganz vielen magischen Dingen eine binäre Dokumentdatei erstellt
    }
}

\clearpage
\printglossary[title=Glossar, toctitle=Glossar]{}
\newpage

    %TC:endignore

    % Sperrvermerk
    \lhead{Sperrvermerk}
    \addsec{Sperrvermerk}
    Die vorliegende Arbeit mit dem Titel \textit{\metaTitel} enthält unternehmensinterne Daten der Firma \textit{Beispiel GmbH}.
    Daher ist sie nur zur Vorlage bei der FOM sowie den Begutachtern der Arbeit bestimmt.
    Für die Öffentlichkeit und dritte Personen darf sie nicht zugänglich sein.
    \newline

    \begin{flushright}
        \includegraphics[width=5cm]{abbildungen/unterschrift.png}
    \end{flushright}
    \begin{minipage}{0.48\textwidth}
        \begin{flushleft}
            \metaOrt{}, \today
        \end{flushleft}
    \end{minipage}
    \begin{minipage}{0.48\textwidth}
        \begin{flushright}
            \metaAutor{}
        \end{flushright}
    \end{minipage}


    % Römische Seitenzahlen werden später weitergezählt
    \newcounter{fort}
    \setcounter{fort}{\value{page}}


    % Hauptteil
    \setcounter{page}{1}
    \pagenumbering{arabic}
    \lhead{\nouppercase{\leftmark}}
    % \include{hauptteil/beispiel}


    % Römische Seitenzahlen fortfahren
    \pagenumbering{Roman}
    \setcounter{page}{\value{fort}}
    %TC:ignore

    % Anhänge
    \appendix
    \setcounter{secnumdepth}{0}
    % \include{anhang/anhang}

    % Literaturverzeichnisse
    \togglefalse{abx@bool@giveninits}
    \printbibliography[heading=bibintoc, nottype=online, title=Literaturverzeichnis]
    \newpage
    \printbibliography[type=online, title=Internetquellen]
    \newpage

    \pagenumbering{gobble}

    % Ehrenwörtliche Erklärung
    \lhead{Ehrenwörtliche Erklärung}
    \section*{Ehrenwörtliche Erklärung}
\phantomsection{}
    Hiermit versichere ich, dass die vorliegende Arbeit von mir selbstständig und ohne unerlaubte Hilfe angefertigt worden ist, insbesondere dass ich alle Stellen, die wörtlich oder annähernd wörtlich aus Veröffentlichungen entnommen sind, durch Zitate als solche gekennzeichnet habe.
    Weiterhin erkläre ich, dass die Arbeit in gleicher oder ähnlicher Form noch keiner Prüfungsbehörde/Prüfungsstelle vorgelegen hat.
    \newline
    Ich erkläre mich nicht damit einverstanden, dass die Arbeit der Öffentlichkeit zugänglich gemacht wird.
    Ich erkläre mich damit einverstanden, dass die Digitalversion dieser Arbeit zwecks Plagiatsprüfung auf die Server externer Anbieter hochgeladen werden darf.
    Die Plagiatsprüfung stellt keine Zurverfügungstellung für die Öffentlichkeit dar.
    \newline

    \begin{flushright}
        \includegraphics[width=5cm]{abbildungen/unterschrift.png}
    \end{flushright}
    \begin{minipage}{0.48\textwidth}
        \begin{flushleft}
            \metaOrt{}, \today
        \end{flushleft}
    \end{minipage}
    \begin{minipage}{0.48\textwidth}
        \begin{flushright}
            \metaAutor{}
        \end{flushright}
    \end{minipage}

    %TC:endignore
\end{document}
